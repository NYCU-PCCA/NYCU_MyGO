%!TEX program = xelatex
\documentclass[a4paper,10pt,twocolumn,oneside]{article}
\setlength{\columnsep}{10pt}    % 兩欄模式的間距
\setlength{\columnseprule}{0pt} % 兩欄模式間格線粗細

\usepackage{xparse}
\usepackage{algpseudocode}
\usepackage{amsthm}             % 定義,例題
\usepackage{amssymb}
\usepackage{amsmath}
%\usepackage[margin=2cm]{geometry}
\usepackage{fontspec}           % 設定字體
\usepackage{setspace}
\usepackage{color}
\usepackage[dvipsnames]{xcolor}
\usepackage{listings}           % 顯示code用的
%\usepackage[Glenn]{fncychap}   % 排版,頁面模板
\usepackage{fancyhdr}           % 設定頁首頁尾
\usepackage{graphicx}           % Graphic
\usepackage{enumerate}
% \setlist[enumerate]{
%   leftmargin=*,
%   nolistsep,
% }
% \setlist[itemize]{
%   leftmargin=*,
%   nolistsep,
% }
\usepackage{array}
\usepackage{tikz}

\usepackage{changepage}
\usepackage[compact]{titlesec}    %compact mode for reducing margin
\usepackage{hyperref}
\hypersetup{
  linktoc=all,
  hidelinks
}

\usepackage[toc]{multitoc}
\renewcommand*{\multicolumntoc}{2}

\usepackage[titles]{tocloft}
\renewcommand\cftsubsecfont{\footnotesize}
\renewcommand\cftsubsecpagefont{\footnotesize}
\renewcommand\cftsecindent{0.1em}
\renewcommand\cftsecnumwidth{1.2em}
\renewcommand\cftsubsecindent{0.5em}
\renewcommand\cftsubsecnumwidth{2.0em}
\renewcommand\cftdotsep{2}
\renewcommand\cftbeforesecskip{0.4em}
\renewcommand\cftbeforesubsecskip{0.2em}

%\usepackage[T1]{fontenc}
\usepackage{courier}
\topmargin=-1pt
\headsep=5pt
\textheight=780pt
\footskip=0pt
\voffset=-40pt
\textwidth=545pt
\marginparsep=0pt
\marginparwidth=0pt
\marginparpush=0pt
\oddsidemargin=0pt
\evensidemargin=0pt
\hoffset=-42pt

\setcounter{secnumdepth}{2}
\setcounter{tocdepth}{2}

% minimize margin
\def\footnotesize{\fontsize{8}{9.5}\selectfont}
\titlespacing*{\section} {0pt}{*0}{*0}
\titlespacing*{\subsection} {0pt}{*0}{*0}
\titlespacing*{\subsubsection} {0pt}{*{-0.5}}{*{-0.5}}

%%%%%%%%%%%%%%%%%%%%%%%%%%%%%

\usepackage[CheckSingle, CJKmath]{xeCJK}

% Consolas, mononoki

\setmainfont{Inconsolata}
\setmonofont{Inconsolata}

% \setCJKmainfont{Noto Sans Mono CJK TC}
% \setCJKmonofont{Noto Sans Mono CJK TC}
\setCJKmainfont[Path=./fonts/]{jf-openhuninn-2.1}  % 中文字型
\setCJKmonofont[Path=./fonts/]{jf-openhuninn-2.1}  % 中文字型

\XeTeXlinebreaklocale "zh"
\XeTeXlinebreakskip = 0pt plus 0pt

%%%%%%%%%%%%%%%%%%%%%%%%%%%%%

\makeatletter
\lst@CCPutMacro\lst@ProcessOther {"2D}{\lst@ttfamily{-{}}{-{}}}
\@empty\z@\@empty
\makeatother
\lstset{
  language=C++,                   % the language of the code
  basicstyle=\footnotesize\ttfamily,  % the size of the fonts that are used for the code
  % basicstyle=\linespread{0.0}\footnotesize\ttfamily,
  %numbers=left,                      % where to put the line-numbers
  %numberstyle=\footnotesize,         % the size of the fonts that are used for the line-numbers
  %stepnumber=1,                      % the step between two line-numbers. If it's 1, each line  will be numbered
  %numbersep=5pt,                     % how far the line-numbers are from the code
  backgroundcolor=\color{white},      % choose the background color. requires \usepackage{color}
  showspaces=false,                   % show spaces adding particular underscores
  showstringspaces=false,             % underline spaces within strings
  showtabs=false,                     % show tabs within strings adding particular underscores
  frame=false,                        % adds a frame around the code
  tabsize=2,                          % sets default tabsize to 2 spaces
  captionpos=b,                       % sets the caption-position to bottom
  breaklines=true,                    % sets automatic line breaking
  breakatwhitespace=false,            % sets if automatic breaks should only happen at whitespace
  escapeinside={\%*}{*)},             % if you want to add a comment within your code
  keywordstyle={\bfseries\color{RoyalBlue}},
  keywordstyle={[2]\bfseries\color{BurntOrange}},
  commentstyle={\itshape\color{BrickRed}},
  stringstyle={\itshape\color{ForestGreen}},
  literate={\ \ }{{\ }}1,
  aboveskip=-0.4em,
  belowskip=-0.4375em,
}

\lstdefinelanguage{C++}
{
  morekeywords={*,size_t,int64_t,ll,ld,pii},    % if you want to add more keywords to the set
  morekeywords={[2]{ALL,eb,chmax,chmin}},
}

\lstdefinelanguage{vim}
{
  % list of keywords
  morekeywords={
    set, se,
    colo, syn,
    let,
    map, nmap, ino, no,
    filetype,
    indent,
    on, off,
    au,
    Plugin,
    call,
    ca,
  },
  morecomment=[l]{"}, % l is for line comment
  morestring=[b]' % defines that strings are enclosed in double quotes
}

\lstdefinelanguage{kotlin}{
  comment=[l]{//},
  emph={filter, first, firstOrNull, forEach, lazy, map, mapNotNull, println},
  emphstyle={\color{OrangeRed}},
  identifierstyle=\color{black},
  keywords={!in, !is, abstract, actual, annotation, as, as?, break, by, catch, class, companion, const, constructor, continue, crossinline, data, delegate, do, dynamic, else, enum, expect, external, false, field, file, final, finally, for, fun, get, if, import, in, infix, init, inline, inner, interface, internal, is, lateinit, noinline, null, object, open, operator, out, override, package, param, private, property, protected, public, receiveris, reified, return, return@, sealed, set, setparam, super, suspend, tailrec, this, throw, true, try, typealias, typeof, val, var, vararg, when, where, while},
  morecomment=[s]{/*}{*/},
  morestring=[b]",
  morestring=[s]{"""*}{*"""},
  morekeywords={[2]{@Deprecated, @JvmField, @JvmName, @JvmOverloads, @JvmStatic, @JvmSynthetic, Array, Byte, Double, Float, Int, Integer, Iterable, Long, Runnable, Short, String, Any, Unit, Nothing}},
  sensitive=true,
}

%%%%%%%%%%%%%%%%%%%%%%%%%%%%%

\ExplSyntaxOn
\NewDocumentCommand{\captureshell}{som}
 {
  \sdaau_captureshell:Ne \l__sdaau_captureshell_out_tl { #3 }
  \IfBooleanT { #1 }
   {% we may need to stringify the result
    \tl_set:Nx \l__sdaau_captureshell_out_tl
     { \tl_to_str:N \l__sdaau_captureshell_out_tl }
   }
  \IfNoValueTF { #2 }
   {
    \tl_use:N \l__sdaau_captureshell_out_tl
   }
   {
    \tl_set_eq:NN #2 \l__sdaau_captureshell_out_tl
   }
 }

\tl_new:N \l__sdaau_captureshell_out_tl

\cs_new_protected:Nn \sdaau_captureshell:Nn
 {
  \sys_get_shell:nnN { #2 } { } #1
  \tl_trim_spaces:N #1 % remove leading and trailing spaces
 }
\cs_generate_variant:Nn \sdaau_captureshell:Nn { Ne }
\ExplSyntaxOff


%%%%%%%%%%%%%%%%%%%%%%%%%%%%%

\begin{document}
\pagestyle{fancy}
\fancyfoot{}
%\fancyfoot[R]{\includegraphics[width=20pt]{ironwood.jpg}}
\fancyhead[L]{NYCU\_MyGO!!!}
\fancyhead[C]{記得初始化 · 確認題目範圍}
\fancyhead[R]{\thepage}
\renewcommand{\headrulewidth}{0.4pt}
\renewcommand{\contentsname}{Contents} 

\begin{spacing}{0.5}
\tableofcontents
\end{spacing}

\scriptsize

%%%%%%%%%%%%%%%%%%%%%%%%%%%%%

\newcommand{\Prefix}{./}
\newcommand{\HashFile}[1]{\captureshell{cpp #1 -dD -P -fpreprocessed | tr -d '[:space:]' | md5sum | cut -c-6}}

\NewDocumentCommand{\IncludeCode}{ O{C++} m m }{
  \subsection[#2]{#2}
  \lstinputlisting[#1]{\Prefix#3}
}

\NewDocumentCommand{\IncludeCodeHash}{ O{C++} m m }{
  \subsection[#2]{#2{ \small[\HashFile{\Prefix#3}] }}
  \lstinputlisting[#1]{\Prefix#3}
}

\newcommand{\IncludeTex}[2]{
  \subsection{#1}
  \input{\Prefix#2}
}

\newcommand{\SectionTitle}[1]{
  \vspace{-0.5em}
  \section{#1}
  \vspace{-1.0em}
}

\SectionTitle{Basic}
\renewcommand\Prefix{../codes/basic/}
\IncludeCode[language=vim]{vimrc}{.vimrc}
\IncludeCode[language=bash]{precompile}{precompile.sh}
\IncludeCode[language=bash]{r.sh}{r.sh}
\IncludeCode[language=bash]{t.sh}{t.sh}
\IncludeCode[]{Default Code}{df.cpp}
\IncludeCodeHash[]{Debug Macro}{debug.cpp}
\IncludeCode[]{Black Magic}{black-magic.cpp}
\IncludeCode[]{Inc Stack}{inc-stack.cpp}
\IncludeCodeHash[]{IO Optimize}{io-optimize.cpp}
\IncludeCode[]{Pragma}{pragma.cpp}


\SectionTitle{Data Structure}
\renewcommand\Prefix{../codes/data-structure/}
% \IncludeCodeHash[]{BIT}{bit.cpp}
\IncludeCodeHash[]{Link Cut Tree}{link-cut-tree.cpp}
% \IncludeCodeHash[]{線段樹}{segment-tree.cpp}
% \IncludeCodeHash[]{Sparse Table}{sparse-table.cpp}
\IncludeCodeHash[]{持久化 Treap}{treap.cpp}
\IncludeCodeHash[]{zkw 線段樹}{zkw-segment-tree.cpp}
\IncludeCodeHash[]{吉如一線段樹}{segment_tree_beats.cpp}


\SectionTitle{Graph}
\renewcommand\Prefix{../codes/graph/}
\IncludeCodeHash[]{2 SAT}{2sat.cpp}
\IncludeCodeHash[]{邊雙連通}{bcc-bridge.cpp}
\IncludeCodeHash[]{點雙連通}{bcc-vertex.cpp}
\IncludeCodeHash[]{數 $C_3$、$C_4$}{count_cycles.cpp}
\IncludeCodeHash[]{重心剖分}{centroid-decomposition.cpp}
\IncludeCodeHash[]{樹鏈剖分}{heavy-light-decomposition.cpp}
\IncludeCodeHash[]{最小有向生成樹 $O(E \cdot f(E))$}{minimum_arborescence-fast.cpp}
% \IncludeCodeHash[]{最小有向生成樹 $O(VE)$}{minimum_arborescence.cpp}
\IncludeCodeHash[]{最小平均環}{minimum_mean_cycle.cpp}
\IncludeCodeHash[]{支配樹}{dominator_tree.cpp}
% \IncludeCodeHash[]{Minimum Steiner Tree}{minimum_steiner_tree.cpp}
\IncludeCodeHash[]{平面圖判定}{planar_check.cpp}
\IncludeCodeHash[]{邊著色 $O(E^2)$}{vizing.cpp}


\SectionTitle{Flow / Matching}
\renewcommand\Prefix{../codes/flow/}
\IncludeCodeHash[]{二分圖匹配(匈牙利)}{match-slow.cpp}
\IncludeCodeHash[]{二分圖匹配}{bipartite_matching.cpp}
\IncludeCodeHash[]{二分圖完美匹配(帶權、KM)}{km.cpp}
\IncludeCodeHash[]{一般圖匹配}{general_matching.cpp}
\IncludeCodeHash[]{一般圖匹配(帶權)}{general_matching_weighted.cpp}
% \IncludeCodeHash[]{Dinic}{dinic.cpp}
\IncludeCodeHash[]{上下界可行流(Dinic)}{bounded_flow.cpp}
% \IncludeCodeHash[]{預流推進}{isap.cpp}
\IncludeCodeHash[]{MCMF(Dijkstra)}{mcmf_dijkstra.cpp}
\IncludeCodeHash[]{MCMF(快速、HLPP)}{hlpp.cpp}
\IncludeCodeHash[]{全域最小割}{global_mincut.cpp}
\IncludeCodeHash[]{最小割樹}{gomory_hu.cpp}
% \IncludeCodeHash[]{這是啥?}{mincost_circulation-xxx.cpp}
\IncludeTex{網路流模型}{model.tex}


\SectionTitle{Math}
\renewcommand\Prefix{../codes/math/}
% \IncludeCodeHash[]{中國剩餘定理}{crt-ckiseki.cpp}
\IncludeCodeHash[]{中國剩餘定理}{crt.cpp}
% \IncludeCodeHash[]{擴展歐幾里德}{exgcd-ckiseki.cpp}
\IncludeCodeHash[]{擴展歐幾里德}{exgcd.cpp}
\IncludeCodeHash[]{快速 GCD}{fast_gcd.hpp}
\IncludeCodeHash[]{floor / ceil Function}{floor-ceil.cpp}
% \IncludeCodeHash[]{約瑟夫問題 $O(n)$}{josephus.cpp}
\IncludeCodeHash[]{約瑟夫問題 $O(k \log n)$}{josephusklogn.cpp}
% \IncludeCodeHash[]{質數線性篩}{linear-sieve.cpp}
% \IncludeTex{Lucas Theorem}{lucas.tex}
\IncludeCodeHash[]{Miller Rabin}{miller-rabin.cpp}
\IncludeCodeHash[]{乘法取模}{modmul.cpp}
\IncludeCodeHash[]{Fraction struct}{fraction.cpp}
\IncludeCodeHash[]{Modular struct}{modular.cpp}
\IncludeCodeHash[]{Pollard Rho}{pollard_rho.cpp}
\IncludeCodeHash[]{Min\_25 篩}{Min_25-warner1129.cpp}
% \IncludeCodeHash[]{大數模板}{bigint.cpp}

\IncludeCodeHash[]{$\min\{k \mid l \le (ak \bmod m) \le r\}$}{mod_min.cpp}
\IncludeCodeHash[]{找線性遞迴關係}{Berlekamp-Massey.cpp}
% \IncludeCodeHash[]{康托展開}{cantor_expansion.cpp}
% \IncludeCodeHash[]{特徵多項式}{char_polynomial.cpp}
\IncludeCodeHash[]{計算矩陣 det}{determinant.cpp}
\IncludeCodeHash[]{離散對數}{discrete_log.cpp}
\IncludeCodeHash[]{$n!$ without factor $p$}{fac_no_p.cpp}
\IncludeCodeHash[]{枚舉 $\text{floor}(x/i)$}{floor_enum.cpp}
% \IncludeCodeHash[]{Floor Sum}{floor_sum.cpp}
\IncludeCodeHash[]{$O($根號$)$ 數質數}{pi_count.cpp}
\IncludeCodeHash[]{二次剩餘}{quadratic_residue.cpp}
% \IncludeCodeHash[]{這是啥?}{schreier_sims.cpp}
% \IncludeCodeHash[]{高斯消去解方程}{simultaneous_equations.cpp}
\IncludeCodeHash[]{Simplex}{simplex.cpp}
\IncludeTex{Simplex 構造}{simplex_construction.tex}

\IncludeCode[]{一些質數}{primes.cpp}
\IncludeTex{各種數字}{numbers.tex}
\IncludeTex{生成函數們}{generating_function.tex}
\IncludeTex{Estimation}{estimation.tex}
\IncludeTex{Euclidean}{euclidean.tex}
\IncludeTex{Theorem}{theorem.tex}


\SectionTitle{Polynomial}
\renewcommand\Prefix{../codes/polynomial/}
% \IncludeCodeHash[]{Fast Fourier Transform}{fft.cpp}
\IncludeCodeHash[]{FFT / NTT}{fft-ckiseki.hpp}
% \IncludeCodeHash[]{Number Theory Transform}{ntt.cpp}
\IncludeCodeHash[]{位元卷積}{fwt.cpp}
\IncludeCodeHash[]{多項式全家桶}{polyop-luogu.cpp}


\SectionTitle{String}
\renewcommand\Prefix{../codes/string/}
\IncludeCodeHash[]{KMP}{kmp.cpp}
\IncludeCodeHash[]{Z Value}{z_value.cpp}
\IncludeCodeHash[]{Manacher}{manacher.cpp}
\IncludeCodeHash[]{最小表示法}{smallest_rotation.cpp}
\IncludeCodeHash[]{SAIS}{sais.cpp}
\IncludeCodeHash[]{DeBruijn Sequence}{de_bruijn.cpp}
\IncludeCodeHash[]{AC 自動機}{ac_automaton.cpp}
\IncludeCodeHash[]{後綴自動機}{sam.cpp}
% \IncludeCodeHash[]{擴展後綴自動機}{ex_sam.cpp}
% \IncludeCodeHash[]{回文樹}{eertree.cpp}


\SectionTitle{Geometry}
\renewcommand\Prefix{../codes/geometry/}
\IncludeCodeHash[]{Default Code}{df.cpp}
\IncludeCodeHash[]{Default Code (old)}{df-old.cpp}
\IncludeCodeHash[]{Closest Pair}{closest-pair.cpp}
\IncludeCodeHash[]{Circle Cover}{circle-cover.cpp}
\IncludeCodeHash[]{Circle Intersect}{circle_interect.cpp}
\IncludeCodeHash[]{Minimum Enclosing Circle}{minimum-enclosing-circle.cpp}


\SectionTitle{Misc}
\renewcommand\Prefix{../codes/misc/}
\IncludeCodeHash[]{Bit Hacks}{bit-hacks.hpp}
\IncludeCodeHash[]{帶修改莫隊}{mos-algorithm-with-modification.cpp}
\IncludeCodeHash[]{擬陣交}{matroid_intersection.cpp}
\IncludeCode[language=python]{Python}{misc.py}
% \IncludeCode[language=java]{Java Example}{example.java}
% \IncludeCode[language=java]{Java BigInt}{big_number.java}
\IncludeCode[language=kotlin]{Kotlin Example}{example.kt}


\SectionTitle{MyGO!!!}
\includegraphics[width=0.5\textwidth]{mygo.png}


\end{document}
